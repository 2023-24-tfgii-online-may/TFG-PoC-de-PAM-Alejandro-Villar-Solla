\capitulo{4}{Técnicas y herramientas}


\section{Metodologías}

Para llevar a cabo este proyecto, se utilizó la metodología de diagrama de Gantt~\cite{gantt}. Esta metodología es  utilizada en la empresa NTT Data en todos los proyectos, y se aplicó de igual manera en la prueba de concepto (POC) de PAM360. La elección de esta metodología se basó en su eficacia para la gestión del tiempo y las tareas, permitiendo una planificación y seguimiento detallado del proyecto. Al utilizar el diagrama de Gantt, se logró una clara visualización de las distintas fases del proyecto, facilitando la identificación de dependencias y el cumplimiento de los plazos establecidos.

\section{Herramientas de desarrollo}

En el desarrollo del proyecto se emplearon diversas herramientas, cada una seleccionada por su idoneidad para las tareas específicas:

\begin{itemize}
	\item \textbf{mRemote}~\cite{mremoteng}: Se utilizó para conectarse al servidor de PAM. Esta herramienta permite gestionar múltiples conexiones de manera eficiente, lo cual fue crucial para acceder y configurar el servidor de PAM360.
	\item \textbf{Habilitación de puertos}: Se habilitó el puerto 8282 para permitir la conexión desde el PC de la empresa al servidor de PAM. Esta configuración fue esencial para asegurar la comunicación entre los sistemas.
	\item \textbf{Windows}: El sistema operativo utilizado para el servidor de PAM y para las operaciones de configuración.
	\item \textbf{Chrome}: Utilizado para acceder a la interfaz web de PAM360, facilitando la administración y configuración del software.
	\item \textbf{Firewall Fortinet}~\cite{fortinet}: Firewall de NTT Data, necesario para conceder permisos de conexión a la máquina de PAM
\end{itemize}

La elección de estas herramientas se basó en su eficacia para la realización las tareas y en las pautas establecidas por la empresa.

\section{Tecnologías y bibliotecas}

En este proyecto, no se utilizaron lenguajes de programación, frameworks o bibliotecas específicos, dado que el enfoque principal fue la configuración y despliegue de PAM360, en lugar del desarrollo de software. 

\section{Comparativas y justificaciones}

Durante la fase inicial del proyecto, se realizaron comparativas entre distintas soluciones de gestión de accesos privilegiados (PAM). Algunas de las soluciones evaluadas incluyeron:

\begin{itemize}
	\item \textbf{CyberArk}~\cite{cyberark}: Conocida por su robustez y funcionalidades avanzadas en la gestión de accesos privilegiados.
	\item \textbf{BeyondTrust}~\cite{beyondtrust}: Reconocida por su enfoque en la seguridad y la gestión de vulnerabilidades.
	\item \textbf{ManageEngine PAM360}: La solución finalmente elegida, destacada por su integración con otros productos de ManageEngine y su relación contractual con NTT Data.
\end{itemize}
	\label{PAM analizados}
\begin{table}[H]
	\centering
	\begin{adjustbox}{max width=\textwidth}
		\begin{tabular}{|l|c|c|c|}
			\hline
			\textbf{Características}                & \textbf{PAM360} & \textbf{BeyondTrust} & \textbf{CyberArk} \\ \hline
			\textbf{Funcionalidades Principales}    & Control de accesos, gestión de contraseñas, auditorías detalladas & Control de accesos, gestión de contraseñas, auditorías detalladas & Control de accesos, gestión de contraseñas, auditorías detalladas \\ \hline
			\textbf{Integración con AD}             & Sí & Sí & Sí \\ \hline
			\textbf{Compatibilidad de Sistemas}     & Windows, Linux, macOS & Windows, Linux, macOS & Windows, Linux, macOS \\ \hline
			\textbf{Implementación}                 & On-premises y Cloud & On-premises y Cloud & On-premises y Cloud \\ \hline
			\textbf{Escalabilidad}                  & Alta & Alta & Alta \\ \hline
			\textbf{Facilidad de Uso}               & Alta & Media & Media \\ \hline
			\textbf{Soporte y Actualizaciones}      & Regular, con opción a soporte premium & Regular, con opción a soporte premium & Regular, con opción a soporte premium \\ \hline
			\textbf{Seguridad y Cumplimiento}       & Cumplimiento de normativas estándar & Cumplimiento de normativas estándar & Cumplimiento de normativas estándar \\ \hline
			\textbf{Costo}                          & Relativamente bajo & Medio-alto & Alto \\ \hline
			\textbf{Características Adicionales}    & Integración con otras soluciones de ManageEngine & Integración con otras soluciones de BeyondTrust & Integración con otras soluciones de CyberArk \\ \hline
		\end{tabular}
	\end{adjustbox}
	\caption{Comparación de PAM360 vs BeyondTrust vs CyberArk}
	\label{tab:comparison}
\end{table}
	
El criterio principal para la selección de PAM360 fue el acuerdo existente entre ManageEngine y NTT Data sobre las cesión de licencias, que facilitó la implementación y soporte de la solución dentro de la empresa. Este acuerdo no solo simplificó el proceso de adquisición, sino que también garantizó una mayor compatibilidad e integración con las herramientas ya utilizadas por NTT Data.


