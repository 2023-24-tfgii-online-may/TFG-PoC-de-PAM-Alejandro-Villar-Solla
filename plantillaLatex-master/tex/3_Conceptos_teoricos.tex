\capitulo{3}{Conceptos teóricos}

Este proyecto requiere entender varios conceptos teóricos fundamentales relacionados con la gestión de accesos privilegiados y las herramientas utilizadas, los cuales explicaré a continuación.

\section{Gestión de Accesos Privilegiados (PAM)}
La gestión de accesos privilegiados (PAM) es una rama de la ciberseguridad que se centra en controlar y supervisar los accesos de los usuarios con privilegios avanzados a los sistemas críticos de una organización. Estos usuarios privilegiados tienen permisos especiales dados por responsables y en todo momento controlados que les permiten acceder a información sensible, modificar configuraciones del sistema y realizar tareas administrativas. Implementar una solución PAM es esencial para prevenir accesos no autorizados, reducir riesgos de seguridad y cumplir con las normativas (Barrett, 2020)~\cite{barrett2020pam} .

\section{Active Directory (AD)}
Active Directory (AD) es un servicio desarrollado por Microsoft, utilizado en redes de dominio de Windows. Su función principal es gestionar y almacenar información sobre los recursos de la red y los usuarios, permitiendo a los administradores gestionar permisos y controlar el acceso a los recursos de la red de manera centralizada. En este proyecto, AD es fundamental, ya que se integra con PAM360 para gestionar los accesos de los usuarios (Microsoft, 2024)~\cite{microsoft2024ad}.

\section{Prueba de Concepto (POC)}
Una Prueba de Concepto (POC) es una implementación preliminar de una solución destinada a demostrar su viabilidad y efectividad. En este trabajo, la POC se centra en desplegar y configurar PAM360 en un entorno controlado, comprobando que cumple con los requisitos y expectativas de la empresa. Esta etapa es crucial para identificar posibles problemas y asegurar que la solución es adecuada antes de una implementación a mayor escala (Smith, 2019)~\cite{smith2019poc}.

\section{PAM360}
PAM360 es una solución integral de gestión de accesos privilegiados desarrollada por ManageEngine. Ofrece funcionalidades avanzadas para controlar, auditar y monitorear el acceso de usuarios privilegiados a recursos sensibles de la empresa. Sus características principales incluyen la gestión de contraseñas, sesiones privilegiadas, auditoría de accesos y generación de informes. PAM360 destaca por su capacidad de integración con diversas plataformas y su facilidad de uso (ManageEngine, 2023)~\cite{manageengine2023pam360}.

\section{Seguridad de la Información}
La seguridad de la información es una disciplina que se ocupa de proteger la información contra accesos no autorizados, modificaciones indebidas y destrucción. Incluye principios fundamentales como la confidencialidad, integridad y disponibilidad. Implementar una solución PAM contribuye correctamente a la seguridad de la información al garantizar que solo los usuarios autorizados puedan acceder a sistemas y datos críticos (Whitman and Mattord, 2022)~\cite{whitman2022security}.

\section{Zero Trust}

El modelo de seguridad Zero Trust se basa en el principio de que las amenazas pueden provenir tanto del interior como del exterior de la red. En lugar de asumir que las entidades dentro de la red son de confianza, Zero Trust requiere verificación continua de identidad y contexto para otorgar acceso. PAM360 facilita la implementación de este modelo al gestionar y supervisar los accesos privilegiados de manera estricta (Kindervag, 2010)~\cite{kindervag2010zero}.

\section{Integración de Sistemas}

La integración de soluciones PAM con otros sistemas de TI, como SIEM (Security Information and Event Management), IAM (Identity and Access Management), y herramientas de monitoreo, es crucial para una gestión de seguridad efectiva. Esta integración permite una visión holística de la seguridad y facilita la detección y respuesta a incidentes (Johnson, 2018)~\cite{johnson2018integration}.

\section{Auditoría y Monitoreo de Seguridad}

La auditoría y el monitoreo continuo son componentes esenciales de una estrategia de ciberseguridad. Registrar y analizar las acciones de los usuarios privilegiados permite detectar actividades sospechosas y responder a incidentes de manera oportuna. PAM360 ofrece capacidades avanzadas de auditoría y generación de informes, proporcionando visibilidad completa sobre el uso de accesos privilegiados (Moore, 2021)~\cite{moore2021monitoring}.

Este apartado ha sintetizado los conceptos teóricos necesarios para comprender el desarrollo del proyecto. La correcta comprensión e implementación de estos conceptos es fundamental para el éxito de la gestión de accesos privilegiados en cualquier organización.