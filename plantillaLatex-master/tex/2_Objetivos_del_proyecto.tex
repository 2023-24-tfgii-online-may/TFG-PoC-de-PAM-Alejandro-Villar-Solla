\capitulo{2}{Objetivos del Proyecto}

El desarrollo de este proyecto tiene como propósito principal implementar y evaluar una Prueba de Concepto (POC) de la solución PAM360 en la infraestructura de NTT Data. Para lograr este propósito, se establecen varios objetivos específicos que abarcan tanto los requisitos funcionales del software como los objetivos técnicos necesarios para llevar a cabo el proyecto de manera satisfactoria. A continuación, se detallan estos objetivos:

\section{Objetivos Funcionales}

\begin{itemize}
	\item \textbf{Despliegue de PAM360:}
	\begin{itemize}
		\item Configurar y desplegar PAM360 en una máquina proporcionada por NTT Data, asegurando que funcione correctamente y esté bien integrada dentro de los parámetros de seguridad de la empresa.
	\end{itemize}
	
	\item \textbf{Gestión de Accesos:}
	\begin{itemize}
		\item Implementar la funcionalidad de PAM360 para gestionar los accesos a las máquinas por parte de los usuarios del Active Directory (AD), tanto internos como clientes, asegurando que todos los accesos sean monitoreados y controlados.
	\end{itemize}
	
	\item \textbf{Seguridad y Control:}
	\begin{itemize}
		\item Asegurar que PAM360 proporcione un control estricto y seguro de los accesos, minimizando el riesgo de accesos no autorizados y mejorando la seguridad interna de la infraestructura de NTT Data.
	\end{itemize}
\end{itemize}

\section{Objetivos Técnicos}

\begin{itemize}
	\item \textbf{Integración con Active Directory:}
	\begin{itemize}
		\item Integrar PAM360 con el sistema de Active Directory (AD) de NTT Data para centralizar la gestión de accesos y facilitar la administración de usuarios.
	\end{itemize}
	
	\item \textbf{Configuración y Personalización:}
	\begin{itemize}
		\item Realizar la configuración inicial de PAM360, ajustando los parámetros y opciones para que se adapten a las necesidades específicas de la empresa y aseguren un funcionamiento óptimo.
	\end{itemize}
	
	\item \textbf{Pruebas de Funcionamiento y Seguridad:}
	\begin{itemize}
		\item Llevar a cabo una serie de pruebas funcionales y de seguridad para verificar que PAM360 opera correctamente, identificando y resolviendo cualquier problema que pueda surgir durante el proceso.
	\end{itemize}
	
	\item \textbf{Documentación del Proceso:}
	\begin{itemize}
		\item Documentar detalladamente cada fase del proyecto, desde la configuración y despliegue hasta las pruebas y resultados, recopilando todos los datos al finalizar.
	\end{itemize}
	
	\item \textbf{Evaluación de Resultados:}
	\begin{itemize}
		\item Evaluar el rendimiento y la efectividad de PAM360 en la gestión de accesos y la mejora de la seguridad, recopilando datos y métricas que permitan analizar su funcionamiento.
	\end{itemize}
\end{itemize}

\section{Objetivos de Mejora y Futuro}

\begin{itemize}
	\item \textbf{Optimización Continua:}
	\begin{itemize}
		\item Identificar posibles mejoras en la configuración y uso de PAM360 para optimizar su rendimiento y adaptabilidad a futuros proyectos que se oferten en la empresa.
	\end{itemize}
	
	\item \textbf{Capacitación y Transferencia de Conocimiento:}
	\begin{itemize}
		\item Asegurar que el equipo de TI de NTT Data esté capacitado para utilizar y mantener PAM360, proporcionando la formación necesaria y transfiriendo el conocimiento adquirido durante el proyecto.
	\end{itemize}
	
	\item \textbf{Propuestas de Mejora:}
	\begin{itemize}
		\item Sugerir posibles mejoras y nuevas funcionalidades para futuras versiones de PAM360, basadas en la experiencia obtenida durante la POC y el feedback propio.
	\end{itemize}
\end{itemize}

Los objetivos de este proyecto no solo buscan implementar y evaluar la funcionalidad de PAM360, sino también asegurar que la solución se integre de manera eficiente y segura en la infraestructura de NTT Data, proporcionando un control efectivo de los accesos y mejorando la seguridad informática de la empresa.
