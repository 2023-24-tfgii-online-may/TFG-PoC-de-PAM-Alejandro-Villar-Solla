\apendice{Plan de Proyecto Software}

\section{Introducción}

Este documento detalla el Plan de Proyecto para la implementación de PAM360 en NTT Data, cubriendo todos los aspectos desde la planificación inicial hasta la finalización del proyecto. El objetivo es proporcionar una guía clara y detallada que asegure el éxito en la implementación y configuración de la herramienta PAM360 para la gestión de accesos privilegiados.

\section{Objetivos}

El objetivo principal de este proyecto es implementar y configurar PAM360 para mejorar la gestión de accesos privilegiados en NTT Data. Los objetivos específicos incluyen:

\begin{itemize}
	\item Mejorar la seguridad mediante el control y monitoreo de accesos privilegiados.
	\item Optimizar la eficiencia en la gestión de usuarios y contraseñas.
	\item Asegurar la transparencia y trazabilidad de las actividades de los usuarios.
	\item Facilitar la auditoría y el cumplimiento normativo.
	\item Integrar PAM360 con las infraestructuras existentes de NTT Data.
\end{itemize}

\section{Alcance del Proyecto}

El alcance del proyecto abarca la instalación, configuración e integración de PAM360 con el Active Directory de NTT Data, incluyendo:

\begin{itemize}
	\item Revisión de documentación técnica.
	\item Preparación del entorno de pruebas.
	\item Despliegue y configuración de PAM360.
	\item Integración con AD y pruebas funcionales.
	\item Configuración de políticas de seguridad y contraseñas.
	\item Generación de reportes y auditorías.
	\item Formación y capacitación para los administradores del sistema.
\end{itemize}

\section{Cronograma}

El proyecto se desarrollará en las siguientes fases:

\begin{itemize}
	\item \textbf{Fase 1: Revisión de Documentación} (1 semana)
	\item \textbf{Fase 2: Preparación del Entorno} (1 semanas)
	\item \textbf{Fase 3: Implementación} (1 semanas)
	\item \textbf{Fase 4: Configuración y Pruebas Funcionales} (1 semanas)
	\item \textbf{Fase 5: Evaluación y Análisis} (1 semanas)
	\item \textbf{Fase 6: Documentación Final y Capacitación} (2 semana)
\end{itemize}


\section{Recursos}

\subsection{Recursos Humanos}

\begin{itemize}
	\item \textbf{Gestor de Proyecto}: Responsable de la coordinación y seguimiento del proyecto.
	\item \textbf{Especialista en Seguridad}: Encargado de la configuración y pruebas de seguridad.
	\item \textbf{Administrador de Sistemas}: Responsable de la integración con AD y gestión de usuarios.
\end{itemize}

\subsection{Recursos Materiales y de Software}

\begin{itemize}
	\item Servidores virtuales para pruebas y despliegue.
	\item Licencias de software para PAM360 y herramientas de virtualización (vSphere, MRemote).
	\item Documentación técnica y manuales de usuario.
	\item Equipos de red y hardware de soporte.
\end{itemize}


\section{Riesgos y Mitigación}

\begin{table}[H]
	\centering
	\begin{adjustbox}{max width=\textwidth}
		\begin{tabular}{|p{0.5\linewidth}|p{0.5\linewidth}|}
			\hline
			\textbf{Riesgo} & \textbf{Mitigación} \\ \hline
			Problemas de Integración con AD & Realizar pruebas preliminares en un entorno controlado. \\ \hline
			Fallos en la Configuración de PAM360 & Contar con el soporte técnico del proveedor y documentación detallada. \\ \hline
			Resistencia al Cambio por parte del Personal & Implementar programas de formación y capacitación. \\ \hline
			Vulnerabilidades de Seguridad & Actualizar regularmente el software y realizar auditorías de seguridad. \\ \hline
			Problemas de Conectividad & Buena configuración en el firewall de permisos y mantenimiento pertinente. \\ \hline
		\end{tabular}
	\end{adjustbox}
	\caption{Riesgos y Mitigación}
	\label{tab:riesgos_mitigacion}
\end{table}

\section{Control y Seguimiento}

Para asegurar el cumplimiento de los objetivos del proyecto, se llevarán a cabo las siguientes actividades de control y seguimiento:

\begin{itemize}
	\item Reuniones semanales de seguimiento con el equipo de proyecto.
	\item Reportes de avance semanales.
	\item Evaluaciones periódicas de cumplimiento de hitos.
	\item Auditorías internas para verificar la correcta implementación y configuración de PAM360.
\end{itemize}


\section{Conclusiones}

El Plan de Proyecto Software para la implementación de PAM360 en NTT Data establece una guía clara para llevar a cabo todas las fases del proyecto, asegurando una gestión eficiente y segura de los accesos privilegiados en la organización. Con una planificación detallada, asignación adecuada de recursos y mitigación de riesgos, se espera alcanzar todos los objetivos propuestos de manera exitosa. Además, la integración con las infraestructuras existentes y la capacitación del personal son factores clave para el éxito del proyecto y su sostenibilidad a largo plazo.


