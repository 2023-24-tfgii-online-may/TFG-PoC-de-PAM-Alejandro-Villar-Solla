\capitulo{1}{Introducción}

Descripción del contenido del trabajo y de la estructura de la memoria y del resto de materiales entregados.

\section{Contexto y motivación}
Hoy en día, la seguridad informática y la gestión de accesos son dos áreas principales que no deben ser subestimadas en una organización, especialmente si se trata de organizaciones que procesan información delicada y requieren la seguridad de la integridad de sus sistemas. NTT Data es una consultoría con una parte de ciberseguridad muy reconocida que se enfrenta al desafío de gestionar sus accesos de una manera segura y eficiente, ya sea de parte de sus empleados u otros clientes externos.

El alto grado de rotación y la necesidad de permitir accesos de manera temporal complican aún más este desafío.

En este contexto, PAM360~\cite{manageengine2023pam360} es una solución de Gestión de Acceso Privilegiado de extremo a extremo que proporciona un control seguro centralizado de todo el acceso a los recursos empresariales críticos. Ayuda a rastrear todos los movimientos y cambios realizados para la seguridad de un usuario.

He elegido esta tarea que he tenido que realizar en NTT Data porque estoy muy interesado en el área de ciberseguridad. Me gustaría trabajar como pentester\footnote{Un pentester es un profesional que realiza pruebas de penetración para identificar y solucionar vulnerabilidades en sistemas de seguridad.} en mi futuro laboral, así que creo que es una gran oportunidad para aprender nuevas tecnologías de seguridad informática observando su desarrollo desde dentro.




\section{Justificación del Proyecto}
La elección de este proyecto es una decisión consensuada con mi tutor en NTT Data, con la sugerencia de implementar una POC para la solución PAM360 en términos de alta seguridad y efectividad en la administración de accesos, ya que la consultora requiere, con sus múltiples clientes y proyectos, un sistema fuerte que soporte la administración de accesos sin comprometer la seguridad.

PAM360 es una solución que puede gestionar y monitorizar los accesos en tiempo real, lo cual es una mejora con respecto a la característica que tenía anteriormente, mediante la cual se solían inscribir a numerosos usuarios dentro del AD de la empresa y estar pendiente de cuándo se expiran cuentas, de los accesos inapropiados, de los cambios de sudoers en linux, etc.

Esta iniciativa está justificada por los beneficios esperados de la implementación de PAM360, que incluyen:
\begin{itemize}
\item Mejora en la seguridad: Al permitir un mejor control y monitoreo de los accesos, se minimiza el riesgo de accesos no autorizados y se garantiza la integridad de los sistemas de la organización.

\item Eficiencia en la gestión de accesos: La centralización de la gestión de accesos permite una administración más sencilla y eficiente, reduciendo la carga de trabajo del equipo de respuesta y prevención.

\item Flexibilidad y escalabilidad: PAM360 permite adaptarse a las necesidades cambiantes de la empresa, facilitando la gestión de usuarios temporales y externos. En caso de que el usuario necesite privilegios adicionales en cualquier sistema, se pueden asignar directamente desde la misma solución en un ambiente controlado. Se pueden asignar o revocar accesos en cualquier momento.

\end{itemize}

\section{Metodología}

Para la realización de este proyecto, se seguirán los siguientes pasos metodológicos:

\begin{itemize}
\item Revisión de documentación: Investigación y revisión de la documentación técnica relacionada con el tema de gestión de acceso privilegiado y PAM360, por ejemplo, los manuales de ManageEngine y las experiencias de los usuarios

\item Planificación y preparación: Establecimiento del alcance del plan de pruebas, preparación y configuración del entorno de pruebas.

\item Implementación: Despliegue y configuración de PAM360, integración con AD y realización de pruebas funcionales y de seguridad.

\item Evaluación y análisis: Análisis de los resultados, evaluación de la eficacia y efectividad de PAM360.

\item Documentación: Redacción del informe final del proyecto, incluyendo la descripción del proceso, resultados obtenidos,capturas, reportes, conclusiones y recomendaciones.

\end{itemize}

\section{Estructura}
Este documento está organizado de la siguiente manera:

\begin{itemize}
\item Introducción: Presenta el contexto, la justificación, los objetivos, la metodología y la estructura del documento.

\item Objetivos del Proyecto: Detalla los objetivos específicos y generales del proyecto.

\item Conceptos Teóricos: Explica los conceptos teóricos necesarios para entender la gestión de accesos privilegiados y la tecnología subyacente en PAM360.

      \item Secciones: Desglosa las diferentes partes del tema.
      \item Referencias: Lista las fuentes y material consultado.
      \item Imágenes: Presenta diagramas y capturas relevantes.
      \item Listas de ítems: Enumera elementos clave y funcionalidades.
      \item Tablas: Ofrece tablas de datos y comparativas.  
\item Técnicas y Herramientas: Describe las técnicas y herramientas utilizadas durante la implementación y pruebas de PAM360.
\item Aspectos Relevantes del Desarrollo del Proyecto: Aborda los desafíos y consideraciones importantes encontrados durante la configuración y pruebas de PAM360.
\item Trabajos Relacionados: Revisa otros trabajos y estudios similares para situar el proyecto en un contexto más amplio.
\item Conclusiones y Líneas de Trabajo Futuras: Presenta las conclusiones del proyecto y sugiere posibles líneas de trabajo futuras para mejorar y expandir la implementación de PAM360 en NTT Data.
\item Anexos: Incluyen el resto de información que no se ha podido incluir en esta memoria
\end{itemize}